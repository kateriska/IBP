% Tento soubor nahraďte vlastním souborem s přílohami (nadpisy níže jsou pouze pro příklad)

% Umístění obsahu paměťového média do příloh je vhodné konzultovat s vedoucím
\chapter{Obsah přiloženého paměťového média}

\chapter{Manuál}
Pro správný běh programu je nutno mít nainstalovaný Python 3.6 a následující knihovny:

\begin{itemize}
    \item \verb=numpy= (knihovna NumPy)
    \item \verb=cv2= (knihovna OpenCV)
    \item \verb=matplotlib=
    \item \verb=skimage= (knihovna scikit-image)
    \item \verb=pywt= (knihovna PyWavelets)
    \item \verb=sklearn= (knihovna scikit-learn)
\end{itemize}

Program se spouští ve tvaru \verb=python3.6 bt.py arguments=, kde \verb=arguments= jsou jednotlivé argumenty uvedené níže s jejich podrobným vysvětlením a funkcí. 

\subsection*{Režim extrakce vektorů a klasifikace}
Program je možné spustit v následujících variantách:\\\\
\verb=-test lbp -s otsu|gauss|mean -clasif ann|svm|dts -img b|g|r|all=\\
Jedná se o extrakci vektorů a následnou klasifikaci na základě vektoru založeném na lokálním binárním vzoru. V tomto případě si uživatel vybere danou metodu segmentace s užitím různých typů prahování (\verb=otsu= pro využití Otsu prahování, \verb=gauss= pro využití adaptivního prahování Gaussian, \verb=mean= pro adaptivní prahování Mean). Dále si uživatel vybere daný princip klasifikace - umělou neuronovou síť \verb=ann=, SVM \verb=svm= nebo rozhodovací stromy \verb=dts=. Poslední a nedílnou součástí je určit, jaké snímky chceme analyzovat - nasvícené pouze modrým světlem (\verb=b=), zeleným (\verb=g=), červeným (\verb=r=), nebo chceme zkoumat všechny snímky získaného datasetu (\verb=all=).\\\\
\verb=-test sobel -clasif ann|svm|dts -img b|g|r|all=\\
Tento příkaz programu je určený pro extrakci a klasifikaci pomocí Sobel a Laplaceova operátoru. Přepínač \verb=-s= tato metoda nevyužívá, segmentace je přímo provedena s využitím prahovací metody Otsu.\\\\
\verb=-test wavelet -t <type> -s otsu|gauss|mean -clasif ann|svm|dts -img b|g|r|all=\\
Každý obraz otisku prstu je analyzován pomocí vlnkové transformace. Získaný vektor může být pomalejší pro získání, zejména u velkých datasetů. Diskrétní vlnková transformace však využívá mnoho různých Wavelet rodin. Přepínač \verb=-t= je určený pro nastavení dané vlnkové rodiny a typu vlnky. Všechny podporované vlnky implementovaného programu \verb=type=, a zároveň i všechny podporované vlnky knihovnou PyWavelets pro 2D diskrétní vlnkovou transformaci, jsou uvedeny v Readme.\\\\
\noindent
Pořadí argumentů není volitelné.

\subsection*{Režim metody zobrazení jednotlivých technik}
Cílem tohoto režimu uživateli snadno předložit různé využité techniky při analýze. Program je možné spustit následovně:\\\\
\verb=-show lbp [-s otsu|gauss|mean] -img <path>=\\
Uživatel proměnnou \verb=path= nahradí cestou k obrazu otisku prstu, který chce analyzovat. Výsledkem je zobrazení otisku prstu zpracovaného pomocí algoritmu LBP a histogram, který byl důležitou vlastností v případě extrakce vektoru pro analýzu.\\\\
\verb=-show sobel [-s otsu|gauss|mean] -img <path>=\\
Tento příkaz umožní zobrazit okno se zpracovaným otiskem prstu pomocí Sobel operátoru na základě osy $x$ a $y$ a výsledku aplikace Laplaceova operátoru.\\\\
\verb=-show wavelet -t <type> [-s otsu|gauss|mean] -img <path>=\\
Provádí vlnkovou transformaci. Výsledkem je horizontální, vertikální a diagonální detail a jejich aproximace.\\\\
\verb=-show seg -s otsu|gauss|mean -img <path>=\\
Zobrazí segmentovaný obraz s využitím daného typu prahování.\\\\
Segmentace je v tomto režimu povinná pouze v posledním případě. Pořadí argumentů není volitelné.

\subsection*{Podporované vlnky}

Knihovna PyWavelets podporuje následující vlnkové rodiny a jejich příslušné vlnky pro diskrétní vlnkovou transformaci:
\begin{itemize}
    \item Biortogonální spline vlnky - bior1.1, \textbf{bior1.3}, \textbf{bior1.5}, bior2.2, \textbf{bior2.4}, bior2.6, bior2.8, bior3.1, bior3.3, bior3.5, bior3.7, bior3.9, bior4.4, bior5.5, bior6.8
    \item Coiflety - coif1, coif2, coif3, coif4, coif5, coif6, coif7, coif8, coif9, coif10, coif11, coif12, coif13, coif14, coif15, coif16, coif17
    \item Vlnky Daubechies - db1, \textbf{db2}, db3, \textbf{db4}, db5, db6, db7, db8, db9, db10, db11, db12, db13, db14, db15, db16, db17, db18, db19, db20, db21, db22, db23, db24, db25, db26, db27, db28, db29, db30, db31, db32, db33, db34, db35, db36, db37, db38
    \item Diskrétní Meyerova vlnka - dmey
    \item Haar vlnka - haar
    \item Reverzní biortogonální spline vlnky - rbio1.1, rbio1.3, rbio1.5, rbio2.2, rbio2.4, rbio2.6, rbio2.8, \textbf{rbio3.1}, rbio3.3, rbio3.5, rbio3.7, rbio3.9, rbio4.4, rbio5.5, rbio6.8
    \item Symlety - sym2, sym3, sym4, sym5, sym6, sym7, sym8, sym9, sym10, sym11, sym12, sym13, sym14, sym15, sym16, sym17, sym18, sym19, sym20
\end{itemize}

Tučně jsou vyznačeny vlnky, se kterým bylo v průběhu práce experimentováno.

%\chapter{Konfigurační soubor}

%\chapter{RelaxNG Schéma konfiguračního souboru}

%\chapter{Plakát}

\chapter{Výsledky analýzy datasetů otisků prstů nasvícených stejnou barvou světla}

\capstartfalse
\begin{table}[!htbp]
\centering
\begin{tabular}{|c|c|c|c|c|}
\hline
\textbf{Segmentace} & \textbf{Klasifikace} & \textbf{Modré světlo} & \textbf{Zelené světlo} & \textbf{Červené světlo} \\ \hline
\textbf{Otsu}       & \textbf{ANN}         & 88.888889             & 96.428571              & 92.592593               \\ \hline
\textbf{}           & \textbf{SVM}         & 92.592593             & 82.142857              & 81.481481               \\ \hline
\textbf{}           & \textbf{DTs}         & 88.888889             & 92.857143              & 92.592593               \\ \hline
\textbf{Gaussian}   & \textbf{ANN}         & 96.296296             & 96.428571              & 100.000000              \\ \hline
\textbf{}           & \textbf{SVM}         & 92.592593             & 57.142857              & 100.000000              \\ \hline
\textbf{}           & \textbf{DTs}         & 88.888889             & 71.428571              & 100.000000              \\ \hline
\textbf{Mean}       & \textbf{ANN}         & 96.296296             & 92.857143              & 100.000000              \\ \hline
\textbf{}           & \textbf{SVM}         & 92.592593             & 71.428571              & 100.000000              \\ \hline
\textbf{}           & \textbf{DTs}         & 85.185185             & 85.714286              & 100.000000              \\ \hline
\end{tabular}
\caption{Srovnání přesnosti v \% pro metodu LBP }
\end{table}
\capstarttrue

\capstartfalse
\begin{table}[!htbp]
\centering
\begin{tabular}{|c|c|c|c|c|}
\hline
\textbf{Segmentace} & \textbf{Klasifikace} & \textbf{Modré světlo} & \textbf{Zelené světlo} & \textbf{Červené světlo} \\ \hline
\textbf{Otsu}       & \textbf{ANN}         & 96.296296             & 92.857143              & 100.000000              \\ \hline
\textbf{}           & \textbf{SVM}         & 92.592593             & 78.571429              & 85.185185               \\ \hline
\textbf{}           & \textbf{DTs}         & 81.481481             & 96.428571              & 88.888889               \\ \hline 
\end{tabular}
\caption{Srovnání přesnosti v \% pro metodu využívající Sobel a Laplaceova operátoru }
\end{table}
\capstarttrue






\capstartfalse
\begin{table}[!htbp]
\centering
\begin{tabular}{|c|c|c|c|c|}
\hline
\textbf{Segmentace} & \textbf{Klasifikace} & \textbf{Modré světlo} & \textbf{Zelené světlo} & \textbf{Červené světlo} \\ \hline
\textbf{Otsu}       & \textbf{ANN}         & 92.592593             & 100.000000             & 88.888889               \\ \hline
\textbf{}           & \textbf{SVM}         & 96.296296             & 85.714286              & 81.481481               \\ \hline
\textbf{}           & \textbf{DTs}         & 92.592593             & 75.000000              & 88.888889               \\ \hline
\textbf{Gaussian}   & \textbf{ANN}         & 88.888889             & 89.285714              & 88.888889               \\ \hline
\textbf{}           & \textbf{SVM}         & 85.185185             & 71.428571              & 92.592593               \\ \hline
\textbf{}           & \textbf{DTs}         & 85.185185             & 78.571429              & 74.074074               \\ \hline
\textbf{Mean}       & \textbf{ANN}         & 85.185185             & 67.857143              & 96.296296               \\ \hline
\textbf{}           & \textbf{SVM}         & 88.888889             & 71.428571              & 88.888889               \\ \hline
\textbf{}           & \textbf{DTs}         & 77.777778             & 75.000000              & 96.296296               \\ \hline
\end{tabular}
\caption{Srovnání přesnosti v \% pro metodu bior1.3 }
\end{table}
\capstarttrue


\capstartfalse
\begin{table}[!htbp]
\centering
\begin{tabular}{|c|c|c|c|c|}
\hline
\textbf{Segmentace} & \textbf{Klasifikace} & \textbf{Modré světlo} & \textbf{Zelené světlo} & \textbf{Červené světlo} \\ \hline
\textbf{Otsu}       & \textbf{ANN}         & 92.592593             & 92.857143              & 96.296296               \\ \hline
\textbf{}           & \textbf{SVM}         & 88.888889             & 85.714286              & 85.185185               \\ \hline
\textbf{}           & \textbf{DTs}         & 77.777778             & 82.142857              & 81.481481               \\ \hline
\textbf{Gaussian}   & \textbf{ANN}         & 74.074074             & 85.714286              & 100.000000              \\ \hline
\textbf{}           & \textbf{SVM}         & 85.185185             & 89.285714              & 81.481481               \\ \hline
\textbf{}           & \textbf{DTs}         & 62.962963             & 78.571429              & 74.074074               \\ \hline
\textbf{Mean}       & \textbf{ANN}         & 74.074074             & 71.428571              & 92.592593               \\ \hline
\textbf{}           & \textbf{SVM}         & 92.592593             & 78.571429              & 81.481481               \\ \hline
\textbf{}           & \textbf{DTs}         & 74.074074             & 82.142857              & 81.481481               \\ \hline
\end{tabular}
\caption{Srovnání přesnosti v \% pro metodu db2 }
\end{table}
\capstarttrue

\capstartfalse
\begin{table}[!htbp]
\centering
\begin{tabular}{|c|c|c|c|c|}
\hline
\textbf{Segmentace} & \textbf{Klasifikace} & \textbf{Modré světlo} & \textbf{Zelené světlo} & \textbf{Červené světlo} \\ \hline
\textbf{Otsu}       & \textbf{ANN}         & 96.296296             & 92.857143              & 96.296296               \\ \hline
\textbf{}           & \textbf{SVM}         & 92.592593             & 82.142857              & 85.185185               \\ \hline
\textbf{}           & \textbf{DTs}         & 92.592593             & 85.714286              & 88.888889               \\ \hline
\textbf{Gaussian}   & \textbf{ANN}         & 92.592593             & 71.428571              & 92.592593               \\ \hline
\textbf{}           & \textbf{SVM}         & 81.481481             & 78.571429              & 88.888889               \\ \hline
\textbf{}           & \textbf{DTs}         & 66.666667             & 92.857143              & 92.592593               \\ \hline
\textbf{Mean}       & \textbf{ANN}         & 92.592593             & 78.571429              & 96.296296               \\ \hline
\textbf{}           & \textbf{SVM}         & 85.185185             & 75.000000              & 96.296296               \\ \hline
\textbf{}           & \textbf{DTs}         & 77.777778             & 60.714286              & 96.296296               \\ \hline
\end{tabular}
\caption{Srovnání přesnosti v \% pro metodu rbio3.1 }
\end{table}
\capstarttrue


\capstartfalse
\begin{table}[!htbp]
\centering
\begin{tabular}{|c|c|c|c|c|}
\hline
\textbf{Segmentace} & \textbf{Klasifikace} & \textbf{Modré světlo} & \textbf{Zelené světlo} & \textbf{Červené světlo} \\ \hline
\textbf{Otsu}       & \textbf{ANN}         & 96.296296             & 92.857143              & 92.592593               \\ \hline
\textbf{}           & \textbf{SVM}         & 88.888889             & 85.714286              & 88.888889               \\ \hline
\textbf{}           & \textbf{DTs}         & 74.074074             & 71.428571              & 74.074074               \\ \hline
\textbf{Gaussian}   & \textbf{ANN}         & 92.592593             & 67.857143              & 100.000000              \\ \hline
\textbf{}           & \textbf{SVM}         & 81.481481             & 60.714286              & 85.185185               \\ \hline
\textbf{}           & \textbf{DTs}         & 81.481481             & 67.857143              & 100.000000              \\ \hline
\textbf{Mean}       & \textbf{ANN}         & 92.592593             & 78.571429              & 96.296296               \\ \hline
\textbf{}           & \textbf{SVM}         & 92.592593             & 78.571429              & 92.592593               \\ \hline
\textbf{}           & \textbf{DTs}         & 88.888889             & 60.714286              & 96.296296               \\ \hline
\end{tabular}
\caption{Srovnání přesnosti v \% pro metodu bior2.4 }
\end{table}
\capstarttrue


\capstartfalse
\begin{table}[!htbp]
\centering
\begin{tabular}{|c|c|c|c|c|}
\hline
\textbf{Segmentace} & \textbf{Klasifikace} & \textbf{Modré světlo} & \textbf{Zelené světlo} & \textbf{Červené světlo} \\ \hline
\textbf{Otsu}       & \textbf{ANN}         & 92.592593             & 96.428571              & 92.592593               \\ \hline
\textbf{}           & \textbf{SVM}         & 96.296296             & 89.285714              & 81.481481               \\ \hline
\textbf{}           & \textbf{DTs}         & 85.185185             & 89.285714              & 92.592593               \\ \hline
\textbf{Gaussian}   & \textbf{ANN}         & 85.185185             & 78.571429              & 88.888889               \\ \hline
\textbf{}           & \textbf{SVM}         & 85.185185             & 71.428571              & 92.592593               \\ \hline
\textbf{}           & \textbf{DTs}         & 77.777778             & 71.428571              & 66.666667               \\ \hline
\textbf{Mean}       & \textbf{ANN}         & 81.481481             & 75.000000              & 96.296296               \\ \hline
\textbf{}           & \textbf{SVM}         & 85.185185             & 75.000000              & 88.888889               \\ \hline
\textbf{}           & \textbf{DTs}         & 88.888889             & 82.142857              & 92.592593               \\ \hline
\end{tabular}
\caption{Srovnání přesnosti v \% pro metodu bior1.5 }
\end{table}
\capstarttrue

\capstartfalse
\begin{table}[!htbp]
\centering
\begin{tabular}{|c|c|c|c|c|}
\hline
\textbf{Segmentace} & \textbf{Klasifikace} & \textbf{Modré světlo} & \textbf{Zelené světlo} & \textbf{Červené světlo} \\ \hline
\textbf{Otsu}       & \textbf{ANN}         & 100.000000            & 92.857143              & 92.592593               \\ \hline
\textbf{}           & \textbf{SVM}         & 88.888889             & 85.714286              & 74.074074               \\ \hline
\textbf{}           & \textbf{DTs}         & 88.888889             & 78.571429              & 92.592593               \\ \hline
\textbf{Gaussian}   & \textbf{ANN}         & 70.370370             & 89.285714              & 85.185185               \\ \hline
\textbf{}           & \textbf{SVM}         & 70.370370             & 71.428571              & 85.185185               \\ \hline
\textbf{}           & \textbf{DTs}         & 70.370370             & 67.857143              & 81.481481               \\ \hline
\textbf{Mean}       & \textbf{ANN}         & 100.000000            & 85.714286              & 100.000000              \\ \hline
\textbf{}           & \textbf{SVM}         & 88.888889             & 89.285714              & 74.074074               \\ \hline
\textbf{}           & \textbf{DTs}         & 96.296296             & 78.571429              & 96.296296               \\ \hline
\end{tabular}
\caption{Srovnání přesnosti v \% pro metodu db4 }
\end{table}
\capstarttrue






\chapter{Výsledky analýzy datasetu obsahující všechny získané snímky otisků prstů}

\capstartfalse
\begin{table}[!htbp]
\centering
\begin{tabular}{|c|c|c|c|}
\hline
\textbf{Segmentace} & \textbf{Klasifikace} & \textbf{LBP} & \textbf{Sobel a Laplaceův operátor} \\ \hline
\textbf{Otsu}       & \textbf{ANN}         & 93.589744    & 97.435897                           \\ \hline
\textbf{}           & \textbf{SVM}         & 88.461538    & 84.615385                           \\ \hline
\textbf{}           & \textbf{DTs}         & 93.589744    & 96.153846                           \\ \hline
\textbf{Gaussian}   & \textbf{ANN}         & 93.589744    &                                     \\ \hline
\textbf{}           & \textbf{SVM}         & 83.333333    &                                     \\ \hline
\textbf{}           & \textbf{DTs}         & 87.179487    &                                     \\ \hline
\textbf{Mean}       & \textbf{ANN}         & 94.871795    &                                     \\ \hline
\textbf{}           & \textbf{SVM}         & 80.769231    &                                     \\ \hline
\textbf{}           & \textbf{DTs}         & 88.461538    &                                     \\ \hline
\end{tabular}
\caption{Srovnání přesnosti v \% pro metody LBP a Sobel a Laplaceův operátor }
\end{table}
\capstarttrue


\capstartfalse
\begin{table}[!htbp]
\centering
\begin{tabular}{|c|c|c|c|}
\hline
\textbf{Segmentace} & \textbf{Klasifikace} & \textbf{bior1.3} & \textbf{db2} \\ \hline
\textbf{Otsu}       & \textbf{ANN}         & 89.743590        & 91.025641    \\ \hline
\textbf{}           & \textbf{SVM}         & 89.743590        & 88.461538    \\ \hline
\textbf{}           & \textbf{DTs}         & 85.897436        & 83.333333    \\ \hline
\textbf{Gaussian}   & \textbf{ANN}         & 87.179487        & 78.205128    \\ \hline
\textbf{}           & \textbf{SVM}         & 88.461538        & 75.641026    \\ \hline
\textbf{}           & \textbf{DTs}         & 82.051282        & 74.358974    \\ \hline
\textbf{Mean}       & \textbf{ANN}         & 88.461538        & 84.615385    \\ \hline
\textbf{}           & \textbf{SVM}         & 80.769231        & 83.333333    \\ \hline
\textbf{}           & \textbf{DTs}         & 84.615385        & 76.923077    \\ \hline
\end{tabular}
\caption{Srovnání přesnosti v \% pro metody bior1.3 a db2 }
\end{table}
\capstarttrue


\capstartfalse
\begin{table}[!htbp]
\centering
\begin{tabular}{|c|c|c|c|}
\hline
\textbf{Segmentace} & \textbf{Klasifikace} & \textbf{rbio3.1} & \textbf{bior2.4} \\ \hline
\textbf{Otsu}       & \textbf{ANN}         & 88.461538        & 93.589744        \\ \hline
\textbf{}           & \textbf{SVM}         & 85.897436        & 88.461538        \\ \hline
\textbf{}           & \textbf{DTs}         & 87.179487        & 83.333333        \\ \hline
\textbf{Gaussian}   & \textbf{ANN}         & 87.179487        & 84.615385        \\ \hline
\textbf{}           & \textbf{SVM}         & 85.897436        & 80.769231        \\ \hline
\textbf{}           & \textbf{DTs}         & 85.897436        & 76.923077        \\ \hline
\textbf{Mean}       & \textbf{ANN}         & 89.743590        & 89.743590        \\ \hline
\textbf{}           & \textbf{SVM}         & 85.897436        & 88.461538        \\ \hline
\textbf{}           & \textbf{DTs}         & 85.897436        & 80.769231        \\ \hline
\end{tabular}
\caption{Srovnání přesnosti v \% pro metody rbio3.1 a bior2.4 }
\end{table}
\capstarttrue






\capstartfalse
\begin{table}[!htbp]
\centering
\begin{tabular}{|c|c|c|c|}
\hline
\textbf{Segmentace} & \textbf{Klasifikace} & \textbf{bior1.5} & \textbf{db4} \\ \hline
\textbf{Otsu}       & \textbf{ANN}         & 89.743590        & 85.897436    \\ \hline
\textbf{}           & \textbf{SVM}         & 87.179487        & 84.615385    \\ \hline
\textbf{}           & \textbf{DTs}         & 85.897436        & 78.205128    \\ \hline
\textbf{Gaussian}   & \textbf{ANN}         & 82.051282        & 88.461538    \\ \hline
\textbf{}           & \textbf{SVM}         & 87.179487        & 85.897436    \\ \hline
\textbf{}           & \textbf{DTs}         & 76.923077        & 78.205128    \\ \hline
\textbf{Mean}       & \textbf{ANN}         & 87.179487        & 87.179487    \\ \hline
\textbf{}           & \textbf{SVM}         & 79.487179        & 87.179487    \\ \hline
\textbf{}           & \textbf{DTs}         & 82.051282        & 85.897436    \\ \hline
\end{tabular}
\caption{Srovnání přesnosti v \% pro metody bior1.5 a db4 }
\end{table}
\capstarttrue



